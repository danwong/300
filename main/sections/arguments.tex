\section{EXTERNAL ARGUMENTS}
\subsection{Encryption: Last Week Tonight with John Oliver (HBO)}
In an influential piece by John Oliver, he argues that whatever happens in this case will have huge rammifications. "Because, the FBI ultimately wants Apple and the entire technology industry to have an encryption always be weak enough that the company can access customer's data if law enforcement needs it." \cite{John-Oliver}.

\subsection{Matt Blaze: A key under the doormat isn't safe. Neither is an encryption backdoor.}
Matt Blaze, an associate professor in the Computer Science Department at the University of Pennsylvania, "studies secure systems cryptography and the impact of technology on public policy." In 1993, the "Clipper Chip" was invented by the NSA and was as a device that would encrypt consumer computer's data but allow officials to access the data if needed. However, Matt Blaze was able to exploit the security flaws in the system. "Clipper's failure starkly demonstrated that cryptographic backdoors must be understood first as a technical problem... Clipper failed not because the NSA was incompetent, but because desigining a system with a backdoor was - and still is - fundamentally in conflict with basic security principles." \cite{Washington-Blaze}

\subsection{Manhattan District Attorney: Smartphone Encryption and Public Safety}
The Manhattan District Attorney's office believes that Apple and technology companies alike are making encryption decisions based on their business interests rather than considering the public's safety interests. "Without legislative action, these corporations will 'continue' to focus on customer and shareholder value,' while government entities 'will try to demonstrate the critical public safety price they (meaning we) pay for 'warrant-proof' platforms'." \cite{Manhattan-da}


\vspace{0.4cm}