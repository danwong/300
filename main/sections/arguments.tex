\section{EXTERNAL ARGUMENTS}
\subsection{Katy Waldman: Facebook's Unethical Experiment}
In an influential Slate article on the study, Katy Waldman argues that the Data Use Policy's vague statements about information usage do not constitute a sufficient enough warning to cover informed consent for the experiment \cite{slate}.

\subsection{Thomas J. Leeper: Science, Social Media, and the Boundaries of Ethical Experimentation}
Thomas Leeper argues that the experiment aligns in a reasonable way with Facebook's interest in improving their product and that the Data Use Policy constitutes informed consent in lieu of explicit notification, as it provides for use of user data for product improvement, and users can opt out by refusing to use the product \cite{leeper}

\subsection{James Grimmelmann: As Flies to Wanton Boys}
The US Code of Federal Regulations specifies the basic elements of informed consent, mandating explanation of the potential risks of participation in a study and the opportunity to refuse to participate \cite{cfrconsent}.  James Grimmelmann argues that Facebook's Data Use Policy does not meet these criteria because it "doesn't even attempt to offer a contact for questions or an opt-out" \cite{laboratorium}.

\subsection{Tal Yarkoni: In Defense of Facebook}
The US Code of Federal Regulations also specifies conditions for waiving the requirements of informed consent that protect the welfare of participants \cite{csrwaive}.  Tal Yarkoni argues that the study satisfies these conditions primarily because of the insignificance and ambiguity of resulting effects \cite{defense} and that the manipulation was “not different from ordinary experience” \cite{defensedefense}.

\subsection{Zeynep Tufekci: Facebook and Engineering the Public}
Zeynep Tufekci expresses concern over Yarkoni's argument and its defense of the status quo, stressing the responsibility of the academic community to "speak up in spite of corporate or government interests" \cite{engineering}.

\subsection{Michelle N. Meyer: Misjudgements will drive social trials underground}
In agreement with Yarkoni, Michelle Meyer highlights the preexisting manipulative nature of Facebook's news feed algorithms.  She questions whether the experiment posed any significant risk to participants and whether the results conclusively demonstrate an increase in negative emotions.  Finally, she warns of the potential consequences of outrage against the experiment \cite{misjudgements}.
\vspace{0.4cm}