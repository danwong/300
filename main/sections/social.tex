\section{SOCIAL IMPLICATIONS}
Whether or not Apple's refusal to create a "backdoor" to unlock the phone of the San Bernardino shooter was ethical, there are numerous important considerations on its impact to the information able to be retrieved on personal phones.\par

When considering the implications of Apple's refusal, there are many concerns about public safety and preventing terrorism. Manhattan district attorney, Cyrus Vance, Jr., says he has 175 iPhones, with potential evidence from serious crimes, including murder, that he wants Apple to aid in opening.\cite{Manhattan-da} Former FBI director, James Comey puts it, "Technology has become a tool of choice for some very dangerous people. Unfortunately, the law has not kept pace with technology and this disconnect has created significant public safety problems we have long described as 'going dark.'"\cite{brookings} Thus, many people consider Apple's refusal to create this "backdoor" to be unjustifiable as it allows dangerous people to protect the information stored on their phone. As Republican Senator Lindsey Graham puts it during the GOP Debate in 2016, "Any system that would allow a terrorist to communicate with somebody inside our country and we can't find out what they're saying, is stupid."\cite{CNN-GOP-Debate}\par

On the contrary, many argue that if this "backdoor" was built, it could lead to huge privacy concerns for the general public. John Oliver explains, "If you penetrate a safe, you have only penetrated that safe. But, a code to open one phone could be modified to work on many more phones."\cite{John-Oliver} Apple's CEO Tim Cook comments, "No one, I believe, would want a master-key built that would turn hundreds of millions of locks even if that key were in the possesion of the person that you trust the most; that key can be stolen... The only way we know to get additional information is to write a piece of software that is the software equivalent of cancer."\cite{ABC-TimCook} Thus, this order and compliance has many important implications regarding overall security of the public including their privacy and preventing terrorism.