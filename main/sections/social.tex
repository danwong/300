\section{SOCIAL IMPLICATIONS}
Whether or not Apple's refusal to create a "backdoor" to unlock the phone of the San Bernardino shooter was ethical, there are numerous important considerations on its impact to the information able to be retrieved on personal phones.\par

When considering the implications of Apple's refusal, there are many criticisms including from FBI Director James Comey. Comey puts it, "Technology has become a tool of choice for some very dangerous people. Unfortunately, the law has not kept pace with technology and this disconnect has created significant public safety problems we have long described as 'going dark.'"\cite{brookings} Thus, many people consider Apple's refusla to create this "backdoor" to be unjustifiable as it allows dangerous people to keep the information on their phone private. As Republican Senator Lindsey Graham puts it during the GOP Debate in 2016, "Any system that would allow a terrorist to communicate with somebody inside our country and we can't find out what they're saying, is stupid."\cite{CNN-GOP-Debate}\par

On the contrary, many argue that if this "backdoor" was built, it could lead to huge privacy concerns for the general public. John Oliver explains, "If you penetrate a safe, you have only penetrated that safe. But, a code to open a phone could be modified many more phones."\cite{John-Oliver} Apple's CEO Tim Cook comments, "No one, I believe, would want a master-key built that would turn hundreds of millions of locks even if that key were in the possesion of the person that you trust the most; that key can be stolen... The only way we know to get additional information is to write a piece of software that is the software equivalent of cancer."\cite{ABC-TimCook} Thus, this order and compliance has many important implications to people's daily lives and security concerns. 






Whether Facebook’s "emotional contagion"
experiment was ethical or
not, there are a number of important
societal considerations to be made regarding
the usage of big data. A major
theme among criticisms of the study
is the difference between standards
for private and academic research.
Cornelius Puschmann, for example,
suggests that "the lines [are] deliberately
being blurred by the quasiacademic
environment cultivated at
major internet companies" \cite{terrain}. This
distinction is important because of
differences in laws surrounding research
in the public and private domains:
private research is not subject
to approval by an institutional review
board (IRB) \cite{irb}. Though the study
involved researchers from the University
of California, San Francisco, and
Cornell University, the data gathered by Facebook
was considered by the researchers a "pre-existing"
dataset that did not need IRB approval \cite{atlantic}. \par
Another important consideration
is the broader context of big data usage
by major tech corporations. Susan
Etlinger questions whether acceptance
of the study would "set a precedent
to use Facebook or other data
to manipulate individuals emotional
states for commercial or other purposes
via 'contagion'" \cite{etlinger}. Others
point out that Facebook is not the
only company to conduct manipulative
experiments, and raise concerns
about the regulation of private research
\cite{guardian}.