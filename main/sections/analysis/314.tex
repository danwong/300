\subsection{Tenet 3.12: Respecting Privacy}
Tenet 3.12 of the Software Engineering Code of Ethics requires software engineers to "\uline{Develop software ... that respect the privacy of those who will be affected by that software}" \cite{code}
\
\subsubsection{Definitions}
\subsubsubsection{Respecting Privacy}
To "respect" privacy is to protect the user's data from being authorized by other individuals. \cite{india}
\subsubsubsection{Actual or Potential Danger}

\subsubsubsection{Affected by Software}
Those who will be affected by the software are generally the users or persons whom have their information stored by that software. Thus, if the information stored by the software became available, it would be a breach of personal privacy of those people.

\subsubsection{Domain Specific Rule}
In the domain of software operating systems, tenet 3.12 requires Apple software engineers to "Develop software that incorporates significant security measures to protect their users' information and data from being stolen." In regards to this case, the operating system that Apple developed uses encryption to protect the information of their iPhone users. 


\subsubsection{Discussion}

\subsubsubsection{Potential Risks}
Encryption is the essential way of protecting data from being accessed without consent. In the letter to customers from Apple, "Smartphones ... have become an essential part of our lives. People use them to store an incredible amount of personal information ... All that information needs to be protected from hackers and criminals who want to access it, steal it, and use it without our knowledge or permission." \cite{appleletter} Apple engineers have a security system in place which encrypts the data on their phones. Thus, the information can not be accessed without entering the correct pin number to unlock the phone. In fact, Apple's security is so strong, in Manhattan District Attorney's office alone, 1445 out of 2000 total Apple iPhones obtained through court-ordered warrants are still locked. \cite{Manhattanda}\par


Apple argues that obliging with the FBI and creating an operating system to weaken the encryption only to be opened by certain individuals would never work. They believe that if this "backdoor" were to be built, then others will find ways to exploit and decrypt private information. "The government suggest this tool could only be used once ... Once the information is known, or a way to bypass the code is revealed, the encryption can be defeated by anyone with that knowledge." \cite{appleletter}\par

In the 1990s, the Clipper Chip was introduced by the National Security Agency. "Clipper Chip, could be used in computers and other devices that needed to encrypt data. But there was a catch: Clipper-encoded data would include a copy of the key used to decrypt it ... If Clipper-encrypted data were encountered during an investigation, the key could be taken out of escrow and the data decrypted." \cite{washingtonBlaze} However, Matt Blaze, along with other security researchers were able to exploit the system. "Clipper failed not because the NSA was incompetent, but because designing a system with a backdoor was - and still is - fundamentally in conflict with basic security principles." \cite{washingtonBlaze} Matt Blaze uses his knowledge with this particular case to argue that creating the "backdoor" requested by the FBI would compromise the safety and integrity of the data. "There is overwhelming consensus in the technical community that even ostensibly 'secure' backdoors put the systems into which they are incorporated at increased risk of outside attack and compromise." \cite{washington-blaze}\par

\subsubsubsection{Respect of Privacy}
The FBI is asking Apple to create a version of the iPhone operating system which would bypass critical security features to aid in recovering the encrypted data on the attacker's phone. "Specifically, the FBI wants us to make a new version of the iPhone operating system, circumventing several important security features, and install it on an iPhone recovered during the investigation."\cite{apple-letter} The FBI is asking Apple to create a "key" that could unlock the specific iPhone. Apple states that it could be modified to open potentially every phone. Apple cites that it could set a dangerous precedent if they complied. "In the wrong hands, this software - which does not exist today - would have the potential to unluock any iPhone in someone's physical possession." \cite{appleletter} The Manhattan District Attorney states that if Apple created this software "backdoor", he would immediately ask Apple to aid his office in unlocking 175 iPhones. Thus, Apple's refusal to the order is due to their concern for users' privacy.

\subsubsection{Conclusion}
Tenet 3.12 of the Software Engineering Code Ethics requires software engineers to respect the privacy of those who will be affected by that software. As such, Apple's operating system encrypts users' information and protects it from becoming breached. As such, Apple was complying with Tenet 3.12. If Apple were to comply with the FBI and create an operating system to allow officials entry to encrypted iPhones, they would in direct conflict with Tenet 3.12. The domain specific rule, as derived from tenet 3.12 of the Software Engineering Code of Ethics, states that Apple should develop software that incorporates significant security measures to protect their users' information and data from being stolen. Because Apple did not build the system requested by the FBI, this rule was satisfied.


\vspace{0.2cm}