\subsection{Tenet 2.03: Knowledge and Consent}
Tenet 2.03 of the SE Code of Ethics requires software engineers to "use the \uline{property} of a \uline{client} ... only in ways \uline{properly authorized}, and with the client's ... \uline{knowledge and consent}" \cite{code}.

\subsubsection{Definitions}
\subsubsubsection{Property}
Digital content consists of "individual files such as images, photos, videos, and text files ... stored either on a device owned by an individual (“locally”), or on devices accessed via the Internet (“in the cloud”), often as part of a service offered by a third party and governed by a contact with the individual", and digital content can be considered "intangible, personal property" \cite{digital-assets}.  User generated content is defined as "published information that an unpaid contributor has provided to a web site", which can include "a photo, video, blog or discussion forum post, poll response or comment made through a social media web site" \cite{define-ugs}.  Because social media posts are user generated content, a form of digital content, and digital content is considered personal property, social media posts can be considered property.  In the context of social media, "property" therefore includes "social media posts".
\subsubsubsection{Client}
A client is "a customer or a person who uses services" \cite{define-client}.  A service "[supplies] public communication" \cite{define-service}.  Social media is defined as a form "of electronic communication... through which users create online communities to share information" \cite{define-social-media}.  Because social media supplies public communication, it can be considered a service.  Therefore, "clients" includes "users" of social media software.
\subsubsubsection{Proper Authorization, Knowledge, and Consent}
To authorize is to "give official permission for or approval to" \cite{define-authorize}, and consent is "permission for something to happen" \cite{define-consent}.  Because authorization and consent are explicit actions, they cannot be properly given without adequate knowledge of the situation, and knowledge can be considered a necessary component of consent and authorization.  Given the overlap between consent and authorization, and the necessity of knowledge in both, "proper authorization, knowledge, and consent" can be simplified to "official approval".

\subsubsection{Domain Specific Rule}
In the domain of social media, tenet 2.03 requires software engineers to
"use the \uline{social media posts} of a \uline{user} only in ways \uline{officially approved} by the \uline{user}."

\subsubsection{Discussion}
As previously mentioned, Facebook researchers claimed that the "emotional contagion" study was conducted in accordance with Facebook's Data Use Policy, which they cite as sufficient informed consent \cite{study}.  In order to justify this claim, the use of user posts in the experiment must be compared with the Data Use Policy, both of which will be examined in the following sections.
\subsubsubsection{Use of Users' Posts}
In what ways did the Facebook researchers use the users' posts? \par 
The "emotional contagion" paper states that user posts were analyzed by the "Linguistic Inquiry and Word Count software (LIWC2007) word counting system" in order to determine whether they were positive or negative \cite{study}.  This data was then used to determine the posts' likelihood of omission from the users' news feeds according to the experimental condition they were assigned to.  No new posts were added to the users' news feeds;  the experiment only involved post omission.  Finally, the paper makes clear that all posts were still accessible via the poster's personal "timeline", and that no personal private messages were affected by the experiment \cite{study}. \par 
To summarize, user posts were analyzed for emotional content and used in the experiment to skew the prevalence of positive or negative posts in participants' news feeds.

\subsubsubsection{What The Data Use Policy Authorized}
What did Facebook users approve of by agreeing to Facebook's Data Use Policy? \par 
In May 2012, Facebook amended its Data Use Policy, adding a line stating that user data may be used "for internal operations, including troubleshooting, data analysis, testing, \textit{research} and service improvement" \cite{forbes}.  Prior to this amendment, which was made \textit{four months} after the "emotional contagion" experiment was conducted, there was no mention of research in the document. \par 
A number of critics of the experiment refer to the updated Data Use Policy without stating that this line did not appear in the document when the experiment was conducted \cite{forbes}.  Thomas Leeper, in his analysis on the ethics of the experiment, argues that agreeing to the policy gives Facebook permission to use user data as long as it is "de-identified", a statement he justifies with the following line from the policy: "we don't share information we receive about you with others unless we have: received your permission; given you notice, such as by telling you about it in this policy; \textit{or} removed your name and any other personally identifying information from it" \cite{leeper}.  While the data was analyzed in an anonymous fashion, Leeper's defense leaps to authorizing Facebook to "use [user] data however [it] want[s]" even though this line only authorizes sharing of anonymous data.  While the previously mentioned line regarding research may have provided such authorization, the latter does not address usage outside of sharing anonymous data, and therefore does not constitute official approval of the experiment by users. 

\subsubsection{Other Means of Authorization}
If the Data Use Policy did not provide official approval at the time of the study, was there some other way Facebook gained approval for the use of user posts in the experiment? \par
One way to gain approval for the experiment would be to request it directly from the participants.  However, as discussed in section 6.1.3.2, Facebook did not provide explicit notification of participation in the experiment.  Even if participants had been notified, though, such notification would only have been relevant to participants' data.  Because the experiment involved filtering posts from the participants' friends, Facebook would have needed approval from non-participants as well.  Because notification of the experiment was not provided to either the participants or their friends (see Section 6.1.3.2), Facebook could not have gotten approval for the experiment from them directly.

\subsubsection{Conclusion}
The researchers behind Facebook's "emotional contagion" study used user data for the purposes of research and experimentation, and they cited Facebook's Data Use Policy as informed consent.  Though this document may have constituted official approval for the use of user posts in the experiment, provisions for use in research were not added until four months after the experiment was conducted; no mention of research was made in the document at the time the experiment was conducted.  In addition, no notification of the experiment was provided to participants or other users whose posts were used for the purposes of the experiment;  the Data Use Policy was the only document through which Facebook attempted to gain approval for the use of user posts.  Because this document did not mention research or experimentation at the time of the study, it did not sufficiently gain this approval.  The experiment therefore used user data in ways not officially approved by the user, and was in violation of the domain specific rule derived above from tenet 2.03 of the Software Engineering Code of Ethics.

\vspace{0.4cm}