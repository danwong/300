
\section{FACTS}
In December 2015, Syed Rizwan Farook and another attacker killed 14 people and seriously injured 22 others. After the attackers died, the FBI was able to recover Farook's work phone. The FBI had the National Security Agency attempt to unlock the phone. However, after a limited amount of incorrect attempts, the phone would automatically delete all of its data. With the NSA's absence of knowledge required to unlock the phone, the FBI turned to Apple and issued valid warrants and subpoenas. Apple complied and gave all of the data and information available to them.\cite{apple-letter}\par

The FBI needs Apple's help beecause the security settings on the iPhone lock may erase all of the phone's data if passwords are entered incorrectly too many times. The FBI requested Apple to engineer an operating system that could be installed onto the attacker's phone to disable critical safety features. This operating system would allow the FBI as many trials to break the 4 digit pin without compromising the phone's encrypted data. \cite{aljazeera}\par

Apple refused the FBI's orders to create an operating system that would circumvent several important security features and to install the operating system on the iPhone recovered during the investigation of the San Bernardino case. Apple believes that building this operating system would create a backdoor and while the government may argue that its use would be limited to this case, there is no way to guarantee such control. \cite{apple-letter}\par

In Apple's letter to their customers, they explain that the "'key' to an encrypted system is a piece of information that unlocks the data, and it is only as secure as the protections around it. Once the information is known, or a way to bypass the code is revealed, the encryption can be defeated by anyone with that knowledge." \cite{apple-letter}\par

Apple and other technology companies alike believe that if Apple complied and created this "backdoor", it could set a very dangerous precedent. 
\vspace{0.4cm}





