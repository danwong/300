\thispagestyle{plain}
\begin{titlepage}
	\begin{center}
    	\Large
    	\textbf{Intentional Weakening of Encryption: The Ethical Implications of Apple's Refusal to Create a "Backdoor"}
    
    	\vspace{0.9cm}
    	\large
    	Daniel Wong
    
    	\vspace{0.4cm}
    	Computer Science
    
    	\vspace{0.4cm}
    	May 2, 2018
    
    	\vspace{0.4cm}
    	CPE 300
    
    	\vspace{1.8cm}
    	\textbf{Abstract}
        
    	\vspace{0.4cm}
       \end{center}
{In December of 2015, two attackers killed 14 people in San Bernardino, California. The attackers destroyed their personal phones but their work iPhones were recovered by the FBI. However, the iPhone required a 4 digit pin to unlock it. The FBI requested data from Apple through valid subpoenas and search warrants. Then, the FBI requested Apple to engineer a version of the iPhone's operating system that would allow it to disable security features once installed. Apple declined this request stating that in the wrong hands, this software can have the potential to unlock any iPhone in someone's physical possesion. Was it ethical for Apple to refuse the FBI's request to create a "backdoor" to all iPhones?} \cite{apple-letter}


\par{The United States goverment urged Apple to comply with the order after being opposed. The FBI stated they would allow Apple to destroy the software once the FBI was able to unlock and remove security features of the attacker's iPhone. Critics argued that Apple and technology companies alike should be held to the same provisions which made cellular encryption weak enough to allow officials to "tap" phone conversations as seen with A5/1. Others argue in defense of Apple stating that the intentional weakening of encrpytion will lead to easy access of the encrypted data. After A5/1 was used to encrypt phone conversations, security researchers were able to attack and easily decrpyt the conversations.}
\end{titlepage}