\section{SOCIAL IMPLICATIONS}
Whether Facebook’s "emotional contagion"
experiment was ethical or
not, there are a number of important
societal considerations to be made regarding
the usage of big data. A major
theme among criticisms of the study
is the difference between standards
for private and academic research.
Cornelius Puschmann, for example,
suggests that "the lines [are] deliberately
being blurred by the quasiacademic
environment cultivated at
major internet companies" \cite{terrain}. This
distinction is important because of
differences in laws surrounding research
in the public and private domains:
private research is not subject
to approval by an institutional review
board (IRB) \cite{irb}. Though the study
involved researchers from the University
of California, San Francisco, and
Cornell University, the data gathered by Facebook
was considered by the researchers a "pre-existing"
dataset that did not need IRB approval \cite{atlantic}. \par
Another important consideration
is the broader context of big data usage
by major tech corporations. Susan
Etlinger questions whether acceptance
of the study would "set a precedent
to use Facebook or other data
to manipulate individuals emotional
states for commercial or other purposes
via 'contagion'" \cite{etlinger}. Others
point out that Facebook is not the
only company to conduct manipulative
experiments, and raise concerns
about the regulation of private research
\cite{guardian}.