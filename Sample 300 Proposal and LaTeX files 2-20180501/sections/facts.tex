
\section{FACTS}
During a week-long period in 2012 (January 11-18), Facebook conducted an experimental study involving deliberate manipulation of users' news feeds in order to study the emotional impact of positive and negative content \cite{study}.  The experiment involved a total of 689,003 users distributed among four experimental groups (reduction of positive or negative posts, and control groups for each type of post reduction).  According to the publication, the researchers wanted to examine whether emotions could be spread via text-based communications that were shared between friends in a non-private context (i.e. "timeline" posts as opposed to private messages).  The study's motivation was the hypothesis that the proportion of positive and negative content included in a user's news feed had a tangible effect on the content they posted themselves.\par

The results of the study demonstrated a statistically significant change in the frequency of positive and negative posts made by the subjects in the experimental groups: "When positive posts were reduced in the News Feed, the percentage of positive words in people’s status updates decreased by B = −0.1\% compared with control... whereas the percentage of words that were negative increased by B = 0.04\%... Conversely, when negative posts were reduced, the percent of words that were negative decreased by B = −0.07\%... and the percentage of words that were positive, conversely, increased by B = 0.06\%" \cite{study}.

Posts were determined to be positive or negative based on the occurrence of "at least one positive or negative word", and because no text was directly seen by researchers, the use of this data was considered to be "consistent with Facebook’s Data Use Policy, to which all users agree prior to creating an account on Facebook, constituting informed consent for this research" \cite{study}. Four months after the study was published, however, Facebook modified its Terms of Service to explicitly mention "research and service improvement" in its handling of user data \cite{forbes}.  The paper states that Facebook's Data Use Policy "constitut[es] informed consent" for the study \cite{study}, and subjects were not debriefed on the experiment after it was completed \cite{atlantic}\par
\vspace{0.4cm}
